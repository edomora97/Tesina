Durante lo sviluppo di un database \emph{relazionale} è bene cercare di rispettare le \emph{forme normali}. Una forma normale è un'insieme di regole che evitano che in una base di dati ci sia  ridondanza o inconsistenza.

Alcune di queste forme normali spesso sono molto restrittive e impediscono lo sviluppo di certi tipi di applicazioni, in particolare la prima forma normale rende pressochè impossibile il \emph{design} di un database flessibile. Per esempio se le entità non sono ben definite a priori o le proprietà di queste sono variabili si raggiunge un limite del modello relazionale.

In questa trattazione si esaminerà uno scenario in cui un semplice database relazionale in prima forma normale non è sufficiente a modellare il problema e si rende necessaria una diversa implementazione.