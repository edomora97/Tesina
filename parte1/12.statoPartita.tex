Ogni partita deve trovarsi in uno stato preciso che indica le condizioni della partita. In particolare è necessario sapere se la partita è già iniziata, se è finita, se ha vinto una certa squadra, se è stata interrotta ecc...

Per rendere più chiaro il valore di questi codici i vari stati delle partite sono stati divisi in alcune fasi: Pre-Partita, In-Partita, Terminata-OK, Terminata-Errore.

\subsubsection{Fase Pre-Partita}
Questa fase è presente solo prima dell'avvio della partita e serve per salvare ed aggiustare i dettagli della partita, come il numero di giocatori, di ruoli, la descrizione, ecc\dots

Questa fase è identificata da codici nell'intervallo:
\[
0 \le x < 100
\]

I codici riconosciuti sono:
\begin{itemize}
	\item \gamestatus{0}{Setup}{Impostazione della partita}
\end{itemize}


\subsubsection{Fase In-Partita}
Questa fase è presente poco prima dell'inizio della partita e durante tutto il corso della partita. 

Viene identificata da codici nell'intervallo:
\[
100 \le x < 200
\]

I codici riconosciuti sono:
\begin{itemize}
	\item \gamestatus{100}{NotStarted}{La partita è attiva e i giocatori possono iniziare ad entrare}
	\item \gamestatus{101}{Running}{La partita è in corso e i giocatori non possono più entrare}
\end{itemize}


\subsubsection{Fase terminata in modo corretto}
Questa fase si verifica quando la partita termina in modo corretto e viene scelta una squadra vincitrice.

I codici che la identificano sono compresi nell'intervallo:
\[
200 \le x < 300
\]

I codici riconosciuti sono:
\begin{itemize}
	\item \gamestatus{200\,+\,y}{Win$y$}{La partita è terminata e ha vinto la squadra $y\ (0\le y < 99)$}
	\item \gamestatus{299}{DeadWin}{La partita è terminata perché tutti i giocatori sono morti}
\end{itemize}



\subsubsection{Fase terminata in modo inaspettato}
Questa fase si verifica quando la partita viene interrotta prima della sua normale fine. In questo caso non viene designato alcun vincitore.

I codici che identificano questa fase sono compresi dall'intervallo:
\[
300 \le x < 400
\]

I codici riconosciuti sono:
\begin{itemize}
	\item \gamestatus{300}{TermByAdmin}{L'amministratore della stanza ha fatto terminare la partita}
	\item \gamestatus{301}{TermBySolitude}{Un numero eccessivo di giocatori hanno abbandonato la partita}
	\item \gamestatus{302}{TermByVote}{\'E stato raggiunto il \emph{quorum} per terminare la partita}
	\item \gamestatus{303}{TermByBug}{Un errore interno del server ha fatto terminare la partita per preservare l'integrità del server}
	\item \gamestatus{304}{TermByGameMaster}{Il GameMaster ha deciso di terminare la partita}
\end{itemize}