Come appena preannunciato il tempo di una partita viene semplicemente memorizzato come numero intero secondo la seguente convenzione:

\[
day
\left\{
\begin{aligned}
\text{se}\ day = 0 						&\Rightarrow \text{arrivo al villaggio} \\
\text{se}\ day\ \text{è \emph{pari}}		&\Rightarrow \text{giorno}\ \frac{day}{2}+1 \\
\text{se}\ day\ \text{è \emph{dispari}}	&\Rightarrow \text{notte}\ \frac{day}{2}+1
\end{aligned}
\right.
\]

Il valore di \emph{day} viene memorizzato nella tabella game in quanto si tratta del giorno della partita.

Sono qui riportati degli esempi dei valori che può assumere \emph{day}:

\begin{tabular}{|c|c|}
	\hline
	\textbf{day} & \textbf{valore} \\
	\hline
	0 & Arrivo al villaggio \\
	1 & Notte 1 \\
	2 & Giorno 2 \\
	3 & Notte 2 \\
	4 & Giorno 3 \\
	5 & Notte 3 \\
	6 & Giorno 4 \\
	7 & Notte 4 \\
	\hline
\end{tabular}