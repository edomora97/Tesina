Lupus in Tabula è un gioco intrinsecamente pieno di polimorfismo, i ruoli ne sono un esempio tipico. Ogni giocatore in una partita possiede un ruolo, possibilmente diverso da quello degli altri giocatori. Ogni tipo di ruolo dispone di proprietà e caratteristiche diverse ed è necessario che il tutto venga gestito in modo molto flessibile per consentire un'aggiunta semplice di nuovi ruoli e funzionalità. 

Alla base della gestione dei vari ruoli c'è una classe base \texttt{Role} che si occupa di istanziare correttamente le classi derivate e la chiamata di eventuali metodi specifici. È necessario prestare una particolare attenzione alle convenzioni usate per gestire i vari ruoli, per esempio il nome del file sorgente è molto importate per la corretta individuazione del ruolo corrispondente.

Ogni ruolo deve personalizzare alcune proprietà statiche derivate dalla classe padre \texttt{Role} per gestire correttamente alcune parti vitali della partita. Per esempio ogni ruolo dispone di una priorità che gestisce l'ordine in cui vengono eseguite le azioni durante la notte.

Le proprità statiche presenti all'interno della classe base \texttt{Role} che è necessario personalizzare sono:

\begin{itemize}
	\item \texttt{\$role\_name} Il nome breve del ruolo, deve essere unico e viene usato per identificarlo. Deve essere composto solo da lettere minuscole dell'alfabeto inglese. Questa proprietà deve anche coincidere con il nome della classe (con la sola iniziale maiuscola) e del file, il quale è nel formato \texttt{class.Ruolo.php}
	\item \texttt{\$name} Nome del ruolo, deve essere \texttt{\$role\_name} con l'iniziale maiuscola
	\item \texttt{\$debug} Se viene impostato a \texttt{true} il ruolo è disponibile sono per gli utenti che hanno accesso alle funzionalità in beta
	\item \texttt{\$enabled} Se viene impostato a \texttt{true} il ruolo non è utilizzabile
	\item \texttt{\$priority} Valore numerico che indica la priorità di esecuzione delle azioni durante la notte. Un ruolo con un indice di priorità inferiore viene eseguito prima
	\item \texttt{\$team\_name} Nome della fazione a cui appartiene il ruolo. Deve essere una delle costanti definite in \texttt{RoleTeam}
	\item \texttt{\$mana} Valore che indica se il ruolo è buono o cattivo. Deve essere una delle costanti definite in \texttt{Mana}
	\item \texttt{\$chat\_groups} Vettore che elenca i codici delle chat disponibili per il ruolo. Per esempio i lupi hanno accesso ad una chat privata. I valori in questo vettore devono essere delle costanti definite in \texttt{ChatGroup}
	\item \texttt{\$gen\_probability} Valore indicativo che definisce un livello di probabilità della generazione del ruolo in caso di generazione automatica dei ruoli. Non è necessario che sia minore di 1
	\item \texttt{\$gen\_number} Numero di copie dello stesso ruolo da generare inseme, per esempio i massoni che vengono generati in coppia avranno questo valore impostato a 2
\end{itemize}

Oltre a queste proprietà statiche che descrivono i vari ruoli è necessaria una parte di codice per l'esecuzione delle varie azioni specifiche di ogni ruolo. La classe base \texttt{Role} dispone di diversi metodi, non solo astratti, da sovrascrivere per personalizzare il comportamento di ogni specifico ruolo. Questi metodi sono qui elencati e descritti, quelli con l'asterisco sono astratti:

\begin{itemize}
	\item \texttt{splash()$^*$} Questa funzione ritorna una stringa da usare come messaggio con il nome del ruolo, per esempio per il ruolo Lupo ritornerà \emph{Sei un lupo}
	\item \texttt{needVoteNight()} Questo metodo di controllo serve per verificare se il giocatore con questo ruolo deve ancora effettuare la votazione notturna. Il valore di ritorno è un vettore che contiene un breve messaggio \texttt{HTML} e una lista degli username votabili. Nella lista degli utenti votabili potrbbero essere presenti delle stringhe che non sono veri username ma dei \emph{segnaposto} come, per esempio, \emph{(nessuno)} per effettuare una votazione nulla. Se la funzione ritorna false il giocatore non deve votare
	\item \texttt{performActionNight()} Viene effettuata l'azione notturna corrispondente al ruolo. Se questa funzione dovesse ritornare \texttt{false} la partita verrebbe interrotta
	\item \texttt{checkVoteNight(\$username)} Controlla se \texttt{\$username} è un valore di votazione valido, se la funzione ritorna false il voto viene annullato
	\item \texttt{needVoteDay()} L'equivalente di \texttt{needVoteNight()} per le azioni diurne
	\item \texttt{performActionDay()} L'equivalente di \texttt{performActionNight()} per le azioni diurne
	\item \texttt{checkVoteDay()} L'equivalente di \texttt{checkVoteNight()} per le azioni diurne
\end{itemize}

Per il funzionamento di alcuni ruoli è anche necessario l'utilizzo di funzioni che permettono il monitoraggio di alcuni parametri della partita. Per esempio tramite le funzioni \texttt{visit} e \texttt{unvisit} è possibile memorizzare che un certo utente ha fatto visita ad un altro, informazione utile per il \emph{paparazzo}. La funzione \texttt{kill} invece si occupa di \emph{uccidere} un giocatore facendo tutti i controlli del caso, per esempio non è possibile uccidere un giocatore morto o un giocatore protetto. È infatti presente un complesso sistema di protezione che si occupa di evitare che alcuni giocatori muoiano per mano di altri. Per esempio è possibile proteggere un giocatore specifico, un intero ruolo o tutti i giocatori da un altro giocatore, un altro ruolo o anche da tutti i giocatori della partita. Questo viene fatto memorizzando una lista di protezioni, le quali vengono identificate da un codice, \texttt{@username} per un utente specifico, \texttt{\#group} per un gruppo, oppure \texttt{*} per ogni giocatore.

All'interno della base di dati, in particolare nella tabella \texttt{player} viene memorizzata l'associazione \texttt{user $\rightarrow$ role}. In particolare come chiave esterna del ruolo viene usata la proprietà \texttt{\$role\_name}. Quando la partita non è ancora inizata e i ruoli non sono ancora stati assegnati ai giocatori viene usato il valore di controllo \texttt{unknown}.

Per istanziare correttamente i ruoli e per ottenere le proprietà statiche richieste viene utilizzata una funzionalità del linguaggio \texttt{PHP} che permette di usare il contenuto di una variabile come nome di classe o di metodo. In particolare cercando di accedere a dei campi statici di una stringa (la quale naturalmente non ne ha) si accede in realtà ai campi di una classe che ha come nome il contentuto della stringa. È necessario però essere certi che il valore della stringa corrisponda esattamente al nome della classe del ruolo e che questa classe derivi da \texttt{Role}. Per questi controlli vengono utilizzate le funzioni standard \texttt{class\_exists} e \texttt{class\_parents}.