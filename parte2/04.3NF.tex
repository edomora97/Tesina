Tra i requisiti che una tabella deve avere per essere in Terza Forma Normale è l'essere in 1NF e in 2NF. Una volta accertato ciò è possibile controllare, ed eventualmente correggere, per la 3NF. Anche questa forma normale è stata inizialmente ideata da Codd nel 1972 in \cite{codd:relationalmodel} ed ha una definizione alquanto complessa: per ogni dipendenza funzionale $A \rightarrow B$, o $A$ è superchiave\footnote{Un insieme di attributi si dice superchiave se contiene interamente la chiave} o $B$ è primo. 

Questa definizione è una semplificazione di quella fornita inizalmente da Codd, la quale si esprime in termini più tecnici, affinchè una relazione sia in terza forma normale è necessario che ogni attributo non-primo sia non-transitivamente dipendente da ogni chiave (\emph{<<every non-prime attribute of R is non-transitively dependent on every key of R>>} \cite{codd:relationalmodel}).

Secondo \textcite{techopedia} la terza forma normale esiste, oltre ad evitare la ridondanza, anche per rendere alcune query più efficienti riducendo in più lo spezio richiesto per la memorizzazione. Si può quindi stabilire che la terza è la forma normale minima consigliata per un qualunque database relazionale, si ha così una buona garanzia di non-ridondanza unita, in generale, a delle buone prestazioni.