La soluzione proposta è alquanto inconsueta, viene infatti implementato un sistema eterogeneo tra database relazionali e NoSQL. Molte delle informazioni da memorizzare sono molto strutturate e i database relazionali sono perfetti per la loro memorizzazione, altri invece non si candidano per essere salvati in un RDBMS. Questi verranno quindi memorizzati in una base di dati esterna a MySQL.

Tra i principali vantaggi di memorizzare i dati in questo modo è la separazione dei dati strutturati da quelli non strutturati. Il database relazionale non dovrà gestire dati complessi, viene infatti mantenuta la prima forma normale. Ovviamente questa soluzione non permette di garantire l'integrità referenziale, non c'è infatti modo di creare un'associazione tra le due basi di dati. 

Sarà quindi necessario avere installate due basi di dati: MySQL e MongoDB. Questi due DBMS non devono necessariamente essere installati nella stessa macchina server, è infatti possibile distribuire il carico in due o più macchine dato che i due database sono indipendenti. 

Naturalmente per evitare problemi di inconsistenza è necessario accertarsi che le transazioni comprendano entrambe le basi di dati, non deve accadere infatti che, per esempio, vengano aggiornati i dati solo nel database relazionale o solo in quello NoSQL. Per accertarsi di ciò è necessario che il codice sia sottoposto a rigidi test.

La dipendenza aggiuntiva di MongoDB può essere un problema per i sistemi datati o limitati. I requisiti di sistema dell'applicazione aumentano dato che è necessario avviare anche un secondo DBMS a fianco di MySQL, è possibile evitare ciò utilizzando un \emph{fallback}. Nel file di configurazione dell'applicazione è possibile specificare che, nel caso in cui MongoDB non fosse disponibile, di usare solo MySQL come base di dati. Il funzionamento di questo metodo è molto semplice, viene utilizzata la prima soluzione [\ref{sec:soluzione1}] cioè i dati da memorizzare come oggetto in MongoDB vengono memorizzati come \texttt{JSON} in MySQL.

L'efficienza di questa soluzione nasce da fatto che MongoDB è pensato proprio per memorizzare dati di questo genere, oggetti senza una struttura costante definita. Sarebbe anche possibile usare solo MongoDB come database per l'intera applicazione ma ci sarebbero notevoli svantaggi: l'applicazione non sarebbe più così portabile, non sarebbe possibile installarla in hosting gratuito per esempio su Altervista. La notevole struttura della maggior parte dei dati dell'applicazione scoraggia l'utilizzo di solo una base di dati non strutturata come un NoSQL.

Un ulteriore punto di forza della coesistenza RDBMS-NoSQL è lo scenario dei \emph{big} dell'informatica. Attualmente alcune tra le più importanti compagnie come Google, Facebook, Digg e Amazon investono molto nello sviluppo di NoSQL\cite{rdbmsnosql} e alcuni dei servizi da loro offerti permettono la coesistenza dei due mondi.