Le \emph{forme normali} sono delle caratteristiche dei database relazionali utili per eliminare delle possibili ridondanze e inconsistenze. La normalizzazione di una base di dati è quindi il processo che si occupa di rendere delle tabelle in forma normale. Esistono numerose forme normali, dalle più semplici alle più complesse.

Il concetto di \emph{dipendenza funzionale} è necessario per comprendere le definizioni delle forme normali dalla terza in poi. Considerando $A$ e $B$ degli insiemi di attributi, la dipendenza funzionale $A \rightarrow B$ indica che, conoscendo $A$, è possibile stabilire il valore di $B$. Per esempio, in una tabella contentente dei dati anagrafici, conoscendo il \emph{codice fiscale} (chiave primaria della tabella) è possibile ricavare il nome e il cognome; si ha quindi la dipendenza funzionale $(CF) \rightarrow (nome,cognome)$.