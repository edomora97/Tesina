Oltre alle quattro principali forme normali ne esistono altre, spesso ignorate e non approfondite in ambito scolastico. Per esempio, dopo la Forma Normale di Boyce-Codd, si può parlare di Quarta Forma Normale introducendo il concetto di \emph{Dipendenze Multivalore}, poi di Quinta e di Sesta. Queste forme normali sono relativamente poco utili e aggiungono un livello notevole di complessità nella base di dati affinché vengano rispettate sempre. Per quasi tutti i casi di database relazionali non è necessario rispettare queste forme normali, limitarsi alla terza o a quella di Boyce-Codd è sufficiente.