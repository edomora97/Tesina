Tutti i dati problematici da memorizzare si possono ridurre, senza perdita di informazione, in una stringa \emph{serializzata}. La serializzazione è quindi il processo che, dati in input dei dati strutturati, ne cambia il formato in qualcosa di memorizzabile o trasmissibile. Esistono diversi metodi che \emph{serializzano} degli oggetti, i due più famosi sono la serializzazione \texttt{XML} e la serializzazione \texttt{JSON}. 

La prima converte l'oggetto in un documento \texttt{XML}, una stringa composta da tag che rappresentano i dati in modo gerarchico in un albero. In modo molto simile anche il \texttt{JSON} modella i dati in una sorta di albero, senza però l'utilizzo di tag ma usando dei simboli come parentesi quadre e graffe. Il principale vantaggio del secondo metodo, a disapito di una migliore strutturazione dei dati, è la semplicità. Un documento \texttt{JSON} è molto più leggero (molti meno byte) e più leggero (più facile da decodificare) rispetto all'equivalente \texttt{XML}.

La prima soluzione è quindi serializzare i dati \emph{problematici} in \texttt{JSON} da memorizzare in colonne dedicate sotto forma di stringhe. Questo però infrange la Prima Forma Normale, esisterebbero infatti degli attributi che non sono atomici dato che sono l'aggregazione di dati semplici.

Dal punto di vista pratico questa soluzione è facilmente realizzabile, ogni volta che è necessario scrivere in quella colonna è suffciente serializzare i dati (per esempio con \texttt{json\_encode}) e ogni volta che vanno letti basta deserializzarli (per esempio con \texttt{json\_decode}).

Il principale problema di questa soluzione, che deriva infatti dalla Prima Forma Normale, è il rischio di inconsistenza. Non è possibile infatti realizzare delle relazioni e tutelare i dati con i vincoli di chiave esterna. È quindi necessario prestare la massima attenzione nella manipolazione dei dati all'interno del database, aggiungendo tutti i controlli del caso lato applicazione.

Questa soluzione è stata utilizzata nelle prime versioni di Lupus in Tabula e solo recentemente è stata modificata a favore della soluzione proposta [\ref{sec:soluzione}].