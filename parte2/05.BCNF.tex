Questa forma normale, spesso chiamata anche 3.5NF, è un'estenzione della terza forma normale. Più propriamente questa forma normale è una restrizione delle condizioni imposte dalla terza, in particolare, oltre al requisito delle 2NF, è necessario che per ogni dipendenza funzionale $A \rightarrow B$, $A$ sia superchiave. Questa limitazione, seppur minima rende alcune relazioni impossibili da ristrutturare, in \cite{wiki:bcnf} sono mostrati alcuni validi esempi di tabelle di questo tipo.

Piccola nota storica: seppur questa forma normale si chiami di \emph{Boyce-Codd}, Chris Date fa notare che già tre anni prima della formulazione dei due informatici, Ian Heath, in una pubblicazione del 1971 \cite{heath:bcnf}, mostrò la stessa definizione. Secondo Date quindi la 3.5NF andrebbe chiamata Forma Normale di Heath ma ciò non viene fatto per la \emph{Legge dell'eponimia di Stigler}\footnote{Questa legge asserisce che: \emph{<<A una scoperta scientifica non si dà mai il nome del suo autore>>} \cite{stigler} }