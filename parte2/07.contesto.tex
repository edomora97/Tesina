In questo capitolo si farà spesso riferimento all'applicazione descritta nella Parte \ref{part:part1}, con particolare riferimento alla sua base di dati.

Come descritto, uno dei problemi principali di Lupus in Tabula è la gestione dei ruoli. I ruoli sono molti e possono aumentare e diminuire molto rapidamente, in più ogni ruolo è intrinsecamente diverso, questi infatti condividono alcune proprietà ma differiscono in altre. Per esempio alcuni ruoli necessitano di memorizzare da qualche parte delle informazioni, spesso molto diverse da quelle memorizzate da qualche altro ruolo. Spesso queste informazioni non sono ben strutturate o a volte lo sono troppo, basti pensare ad un ruolo che deve memorizzare delle informazioni come una lista di giocatori o, peggio ancora, un albero di dati.

I ruoli non sono l'unico caso di problematica nella base di dati, ci sono altre sezioni dell'applicazione che soffrono di problematiche simili. La parte di generazione di una partita per esempio deve memorizzare alcune informazioni per generare i ruoli dei giocatori. Queste informazioni sono mantenute in un oggetto composto da alcune liste che, per esempio, indicano per ogni ruolo possibile il numero di giocatori da generare.

Un'altra problematica è la memorizzazione di alcune informazioni riguardo ai messaggi letti nelle chat. Ogni giocatore ha accesso ad un numero variabile di chat nelle quali possono arrivare dei messaggi che l'utente può leggere. Per sapere quali messaggi deve ancora leggere è necessario sapere fino a dove è arrivato con la lettura. È quindi necessario memorizzare queste informazioni in qualche posto, la problematica è molto simile a quella per la generazione della partita, ogni possibile chat deve sapere l'ultimo messaggio letto (o il relativo \texttt{timestamp}).

Infine la quarta parte che soffre di questo problema riguarda gli eventi della partita. Durante il gioco accadono degli eventi che possono essere di molti tipi diversi. Per esempio l'inizio della partita genera un evento con la data di inizio, la morte di un giocatore ne genera un'altro con lo username del giocatore morto, l'azione di un veggente ne genera un altro con i dati della visione. Man mano che il numero di ruoli aumenta il numero di eventi possibili cresce. La problematica è quindi identica a quella relativa alla memorizzazione delle informazioni dei ruoli.

Tutte queste problematiche sono risolvibili con la stessa strategia dato che, in generale, sono raggruppabili in uno stesso problema comune: la memorizzazione di dati complessi e con una struttura variabile.