La Seconda Forma Normale (2NF) fu anch'essa stilata da E. Codd nel 1971 \cite{codd:relationalmodel}. La 2NF è un'estensione della prima forma normale; per essere in seconda forma normale, quindi, la tabella deve anche essere in 1NF.

Secondo la definizione, ogni attributo non primo\footnote{Un attributo si dice primo quando fa parte di una chiave candidata} deve dipendere dall'intera chiave candidata e non da parte di essa. Per esempio, in una relazione, $R(\underline{A,B},C,D,E)$ dove le dipendenze funzionali sono $(A,B) \rightarrow (C,D,E)$ e $(A) \rightarrow (C)$, quindi $A$ e $B$ compongono la chiave. La relazione non è in seconda forma normale in quanto l'attributo $C$ dipende solo da $A$, quindi da parte della chiave.

Questo è esempio pratico \cite{book:eprogram} di tabella che non è in seconda forma normale, la tabella memorizza il numero di reti di ogni giocatore in un determinato anno, in più viene memorizzato anche il luogo di nascita di ogni giocatore.

\begin{table}
	\centering
	\caption{Tabella campionati}
	\begin{tabular}{llll}
		\textbf{\underline{nome}}                     & \textbf{luogo}               & \textbf{\underline{anno}}                  & \textbf{reti}          \\ \hline
		\multicolumn{1}{|l|}{Rossi Mario} & \multicolumn{1}{l|}{Firenze} & \multicolumn{1}{l|}{1998/1999} & \multicolumn{1}{l|}{3} \\ \hline
		\multicolumn{1}{|l|}{Rossi Mario} & \multicolumn{1}{l|}{Firenze} & \multicolumn{1}{l|}{1999/2000} & \multicolumn{1}{l|}{4} \\ \hline
		\multicolumn{1}{|l|}{Conti Bruno} & \multicolumn{1}{l|}{Roma}    & \multicolumn{1}{l|}{1998/1999} & \multicolumn{1}{l|}{6} \\ \hline
	\end{tabular}	
\end{table}

Per risolvere il problema e rendere la tabella in (almeno) seconda forma normale è possibile dividere la relazione in due, creando una tabella per i giocatori e una per i campionati. In particolare, dato che il luogo dipende solo dal nome del giocatore è possibile memorizzare questa informazione una volta sola in una tabella separata.

\begin{table}
	\centering
	\caption{Soluzione: tabella giocatori e tabella campionati}
	\begin{tabular}{cc}
		\begin{minipage}{.5\linewidth}
			\begin{tabular}{ll}
				\textbf{\underline{nome}}                     & \textbf{luogo} \\ \hline
				\multicolumn{1}{|l|}{Rossi Mario} & \multicolumn{1}{l|}{Firenze} \\ \hline
				\multicolumn{1}{|l|}{Conti Bruno} & \multicolumn{1}{l|}{Roma} \\ \hline
			\end{tabular}
		\end{minipage}
		
		\begin{minipage}{.5\linewidth}
			\begin{tabular}{lll}
				\textbf{\underline{nome}}                     & \textbf{\underline{anno}}                  & \textbf{reti}          \\ \hline
				\multicolumn{1}{|l|}{Rossi Mario} & \multicolumn{1}{l|}{1998/1999} & \multicolumn{1}{l|}{3} \\ \hline
				\multicolumn{1}{|l|}{Rossi Mario} & \multicolumn{1}{l|}{1999/2000} & \multicolumn{1}{l|}{4} \\ \hline
				\multicolumn{1}{|l|}{Conti Bruno} & \multicolumn{1}{l|}{1998/1999} & \multicolumn{1}{l|}{6} \\ \hline
			\end{tabular}
		\end{minipage}
	\end{tabular}
\end{table}