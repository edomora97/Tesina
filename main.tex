\documentclass[b5paper,10pt,twoside,cucitura]{toptesi}
\usepackage{lipsum}
\usepackage{hyperref}
\usepackage{tabularx}
\usepackage{framed}

\hypersetup{%
    pdfpagemode={UseOutlines},
    bookmarksopen,
    pdfstartview={FitH},
    colorlinks,
    linkcolor={blue},
    citecolor={red},
    urlcolor={blue}
  }

\usepackage[utf8]{inputenc}
\usepackage[T1]{fontenc}\usepackage{lmodern}

\ateneo{ITST J.F. Kennedy}
\nomeateneo{Pordenone}
\FacoltaDi{Specializzazione Informatica}
\titolo{Lupus in Tabula}
\sottotitolo{Limiti della Prima Forma Normale e NoSQL}

\candidato{\tabular{@{}l@{}}Edoardo \textsc{Morassutto}\\classe: Quinta C IA\endtabular}

\sedutadilaurea{\textsc{Anno~scolastico} 2015-2016}
\logosede{lion}

\newcommand{\role}[9]{
\begin{framed}
	\def\arraystretch{0.8}
	\begin{tabularx}{0.75\textwidth}{rX rl}
		ID			& #1 & Priorità		& #5 \\
		Nome		& #2 & Debug		& #6 \\
		Mana		& #3 & Enabled		& #7 \\
		Squadra		& #4 & Chat			& #8 \\
	\end{tabularx}
	%\vspace*{0.07cm}
	\\
	#9
\end{framed}
}

\newcommand{\gamestatus}[3]{
	\textbf{#1}: \textbf{\texttt{#2}}$\quad$ #3
}

\newcommand{\apicode}[3]{
	\texttt{#1} & \texttt{#2} & #3 \\
}

\newcommand{\eventcode}[4]{
	\textbf{#1}: \textbf{\texttt{#2}}$\quad$ #3 
	\vspace{0.1cm}
	\\
	Dati registrati dell'evento:\\
	\begin{tabular}{ll}
		#4
	\end{tabular}
	\vspace{0.2cm}
}

\begin{document}

\frontespizio

\sommario
\emph{Lupus in Tabula} è un gioco di ruolo complesso nel quale i giocatori interpretano dei personaggi e devono raccontare una storia cercando di far vincere la propria squadra. Il gioco, spesso noto anche come Mafia\cite{wiki:mafia}, si basa sul riuscire a dedurre il ruolo degli altri giocatori per capire da chi è composta la propria squadra. 

In questo documento è stato analizzato il problema di modellare i dati delle partite e di creare un server che permettesse a molti utenti di giocare in contemporanea. Durante lo sviluppo ci sono state delle difficoltà riguardo la memorizzazione di alcuni dati delle partite che sono stati in parte risolti.

Il documento è stato diviso in due parti, la prima è un'analisi completa dell'applicazione trattando nel dettaglio ogni aspetto sia progettuale che implementativo, la seconda invece, dopo un'introduzione alle forme normali, tratta delle problematiche incontrate e di alcune delle possibili soluzioni.

Durante lo sviluppo di un database \emph{relazionale} è bene cercare di rispettare le \emph{forme normali}. Una forma normale è un'insieme di regole che evitano che in una base di dati ci sia  ridondanza o inconsistenza.

In questa trattazione si esaminerà uno scenario in cui un semplice database relazionale in prima forma normale non è sufficiente a modellare il problema e si rende necessaria una diversa implementazione.

L'applicazione è stata completata ed è disponibile su \url{https://lupus.serben.tk}, tutto il sorgente è pubblico ed accessibile su \url{https://github.com/lupus-dev/lupus}.

\indici

\mainmatter

%%%%%%%%%%%%%%%%%%%%%%%%%%%%%%%%%%%%%%%%%%%%%%%%%%%%%%%
%                  PARTE UNO                          %
%%%%%%%%%%%%%%%%%%%%%%%%%%%%%%%%%%%%%%%%%%%%%%%%%%%%%%%
\part{Lupus in Tabula}


\chapter{Introduzione generale}

\section{Descrizione del gioco}
ciao mondo
Lupus in Tabula è un gioco di ruolo per, normalmente, 7 o più giocatori. Il narratore, che giocando senza computer, è un giocatore speciale che racconta una storia. La storia viene sviluppata da tutti i giocatori che, compiendo delle azioni, comportano dei cambiamenti alla vicenda, per esempio uccidendo dei personaggi o facendoli risorgere.

I giocatori sono divisi in due o più gruppi, esistono sempre le fazioni dei \emph{lupi} e dei \emph{contadini} alle quali può aggiungersi quella dei \emph{criceti mannari} ed altre. Quando la partita finisce solo una di queste squadre vince.

Lo scopo della squadra dei \emph{lupi} è quello di sbranare tutti gli altri giocatori, mentre quello dei \emph{contadini} è di individuare ed uccidere tutti i \emph{lupi}.

La partita si alterna di giorni e di notti, durante ogni notte tutta la squadra dei lupi deve scegliere un bersaglio (solitamente un contadino) che quella notte verrà sbranato. Durante il giorno, invece, ogni giocatore deve votare chi, secondo lui, mandare al rogo. Questo è l'unico modo per i contadini di uccidere i vari lupi che si aggirano nel villaggio. A questa votazione partecipano anche i lupi che cercheranno ovviamente di non farsi votare.

Per rendere il gioco più dinamico vengono inseriti tra i giocatori alcuni ruoli speciali. Per esempio un \emph{veggente} è un contadino che, durante la notte, può scegliere un giocatore vivo e, tramite consultazione della palla di cristallo, sapere se ha un'anima buona oppure cattiva. Ogni ruolo infatti, oltre ad essere di una certa squadra, ha anche delle caratteristiche extra come il \emph{mana}, che può essere buono oppure cattivo. Normalmente (ma non necessariamente) i giocatori nella squadra dei lupi hanno mana cattivo mentre quelli nella squadra dei contadini ce l'hanno buono.

Ogni giocatore, in generale, conosce solo il proprio ruolo e deve dedurre quello degli altri giocatori; ci possono essere dei casi in cui qualche giocatore conosce il ruolo esatto di qualcun altro. Per esempio ogni lupo conosce esattamente quali sono i suoi compagni di squadra.

Quanto un giocatore muore in una partita normale è il narratore ad annunciarlo, in questa implementazione è presente un giornale nel quale viene scritto il nome dei giocatori che sono morti.

Affinchè una partita termini correttamente deve verificarsi alemno una delle seguenti condizioni:

\begin{itemize}
	\item Tutti i giocatori sono morti
	\item Almeno una fazione ha vinto
	\begin{itemize}
		\item Lupi: Il numero di lupi è maggiore o uguale al numero di contadini
		\item Contadini: Il numero di lupi è zero
	\end{itemize}
\end{itemize}

\section{Ruoli}
Ad ogni giocatore viene assegnato dal sistema uno ed un solo ruolo, quindi il giocatore saprà anche la sua fazione e il suo mana. 

Ogni ruolo, oltre al nome ha anche alcune proprietà aggiuntive:

\begin{itemize}
	\item Mana: indica se il giocatore agirà come malintenzionato oppure come una brava persona
	\item Nome identificativo: nome unico che identifica il ruolo
	\item Priorità: serve per stabilire un ordine nell'esecuzione delle azioni
	\item Debug: alcuni ruoli sono accessibili solo dagli sviluppatori e sono marcati come \emph{debug}
	\item Enabled: È possibile disattivare alcuni ruoli con questa proprietà
	\item Chat: alcuni ruoli hanno dei canali di comunicazione dedicati
	\item Gen*: parametri aggiuntivi per la generazione automatica dei ruoli, come numero minimo e massimo di occorrenze del ruolo, probabilità, ecc...
\end{itemize}

\subsection{Ruoli comuni}

I ruoli che normalmente vengono usati in una partita di Lupus in Tabula sono:

\role{Lupo}{Lupo}{Cattivo}{Lupi}{100}{no}{sì}{Lupi}{Durante la notte i \emph{lupi} votano chi eliminare, se almeno il 50\%+1 dei \emph{lupi} vivi votano la stessa persona, questa è una candidata a morire}

\role{Guardia}{Guardia}{Buono}{Contadini}{10000}{no}{si}{}{Durante la notte la \emph{guardia} può scegliere di proteggere una persona, se i \emph{lupi} quella notte decidessero di ucciderla, essa non muore}

\role{Medium}{Medium}{Buono}{Contadini}{150}{no}{si}{}{Il \emph{medium} durante la notte può scegliere di guardare un giocatore morto, lui saprà se quel giocatore aveva un mana buono o cattivo}

\role{Paparazzo}{Paparazzo}{Buono}{Contadini}{1000}{no}{si}{}{Il \emph{paparazzo} durante la notte sceglie una persona da pedinare, vengono riportati sul giornale della mattina seguente tutti i giocatori che hanno visitato il personaggio \textsl{paparazzato}}

\role{Criceto mannaro}{Criceto mannaro}{Cattivo}{Criceti}{10000}{no}{si}{}{\'E un giocatore normale, senza poteri speciali. Se la partita termina e lui è ancora vivo allora vince solo lui e non la sua fazione}

\role{Assassino}{Assassino}{Cattivo}{Contadini}{10}{no}{si}{}{L'\emph{assassino} una sola volta nella partita può scegliere una persona e ucciderla}

\role{Massone}{Massone}{Buono}{Contadini}{10000}{no}{si}{Massoni}{I \emph{massoni} non hanno poteri però hanno una chat dedicata e quindi si conoscono tra loro}

\role{Contadino}{Contadino}{Buono}{Contadini}{10000}{no}{si}{}{I \emph{contadini} non hanno poteri\dots}

\role{Pastore}{Pastore}{Buono}{Contadini}{50}{sì}{si}{}{I \emph{pastori} possono scegliere di sacrificare delle pecore per salvare dei giocatori dalle grinfie dei lupi}

\role{Sindaco}{Sindaco}{Buono}{Contadini}{10000}{no}{si}{}{Il \emph{sindaco} è un contadino che non può essere messo al rogo nella votazione diurn}

\chapter{Analisi}

\section{Analisi del problema}
È necessario sviluppare un sistema robusto che gestisce gli utenti e le partite di Lupus in Tabula.

Il sistema deve essere in grado di sopportare un numero crescente di giocatori e un numero molto elevato di partite, deve essere in grado di scalare ottimalmente con le richieste. La base di dati deve essere sviluppata in modo da garantire un funzionamento efficiente e una robustezza dei dati.

I dati identificativi degli utenti devono essere conservati in modo sicuro, è necessario proteggere le credenziali di accesso tramite avanzati sistemi di \emph{hashing}.

È necessario fare molta attenzione ai privilegi degli utenti, proteggendo le risorse che non sono accessibili. Per esempio è opportuno evitare che gli utenti possano vedere, entrare o modificare le partite alle quali gli è stato negato il permesso.

Le chiavi interne del database non sono visibili agli utenti, delle chiavi alternative non numeriche sono visualizzate dagli utenti. Per esempio le partite non vengono identificate (lato utente) da un intero progressivo ma da un \emph{nome breve}, una sequenza di caratteri che rispetta la seguente \texttt{regex}: \texttt{[a-zA-Z][a-zA-Z0-9]\{0,9\}}. Deve essere lungo da 1 a 10 caratteri \texttt{ASCII}, il primo carattere deve essere una lettera e deve essere unico nel suo contesto.

\section{Stuttura dei componenti}
% Definizione di stanza, partita ecc...

\section{Progettazione della base di dati}
\subsection{Schema concettuale}
\subsection{Schema logico}
\subsection{Descrizione delle entità}
\subsection{Dizionario degli attributi}

\section{Componenti del gioco}
% descrizione del server e delle varie teconologie usate per sviluppare

\section{Codici utilizzati}
\subsection{Tempo del gioco}
\subsection{Stato della partita}
\subsection{Codici di risposta delle API}
\subsection{Codici degli eventi}

\chapter{Implementazione}

\section{Tecnologie utilizzate}
\section{Pagine web}
\section{Webservice REST}

%%%%%%%%%%%%%%%%%%%%%%%%%%%%%%%%%%%%%%%%%%%%%%%%%%%%%%%
%                  PARTE DUE                          %
%%%%%%%%%%%%%%%%%%%%%%%%%%%%%%%%%%%%%%%%%%%%%%%%%%%%%%%
\part{Limiti della 1NF}

\chapter{Introduzione alle forme normali}

\section{Prima forma normale}
\section{Seconda forma normale}
\section{Terza forma normale}
\section{Forma normale di Boyce-Codd}
\section{Quarta e Quinta forma normale}

\chapter{Problematiche con la 1NF}

\section{Contesto del problema}

\section{Prime soluzioni parziali}
\subsection{Una tabella per ruolo}
\subsection{Una tabella key-value}

\section{Soluzione proposta}

\end{document}
